% The UCW Macro Collection: Destinations and links
% Written by Martin Mares <mj@ucw.cz> in 2018 and placed into public domain
% -------------------------------------------------------------------------

\ucwdefmodule{link}

% Should clickable links be produced?
\newif\ifclickable
\ifpdf
	\clickabletrue
\else
	\clickablefalse
\fi

% Common style of all clickable links
\pdflinkmargin=1pt
\def\commonlinkargs{height \the\dimexpr\ht\strutbox-0.5pt\relax depth \the\dimexpr\dp\strutbox-0.5pt\relax attr {/C [0 0 0.5] /Border [0 0 2]}}

% Define a PDF destination for the current position at the page
\def\destpos#1{\ifclickable\pdfdest name {#1} xyz\relax\fi}

% Define a PDF destination for the current page
\def\destpage#1{\ifclickable\pdfdest name {#1} fit\relax\fi}

% Typeset a clickable link to the given destination
% \link{dest}{text}
\def\link#1#2{%
	\ifclickable
		\pdfstartlink\commonlinkargs goto name {#1}\relax
		#2%
		\pdfendlink\relax
	\else
		#2%
	\fi
}

% Typeset a clickable link to the given page number
% (This does not use named destinations. We use it in tables of contents and indices,
% where absolute page numbers are known from other sources.)
% \linkpage{page}{text}
\def\linkpage#1#2{\ifclickable\pdfstartlink\commonlinkargs goto page #1 {/Fit}\relax #2\pdfendlink\else #2\fi}

% Typesetting of URLs is tricky:
%
%   - They can contain various characters considered special by TeX.
%   - We want to adjust appearance of "//", "_", "~" according to font.
%   - "--" should not produce a ligature.
%   - We want to insert a breakpoint after "/", "?", "&".
%   - We need the raw form of the URL for PDF links.
%   - We cannot rely purely on changing catcodes, as we sometimes need
%     to parse URLs given as arguments of macros.
%   - Sometimes, it is useful to insert a manual line break or another
%     typesetting hack to the URL.
%
% Therefore:
%
%   - In our front-end macros (\url, \linkurl) we switch catcodes
%     to accept '%' and '#' as normal characters; if the macros are
%     called indirectly, these characters must be escaped as '\%' and '\#'.
%   - The URL is preprocessed: special characters (with their original
%     catcode) are replaced by calls of auxiliary macros.
%   - When producing PDF links, the auxiliary macros expand to ordinary
%     ASCII characters.
%   - When typesetting the URL, the auxiliary macros expand differently.
%     Furthermore, they can be temporarily re-defined in the \urlprefix macro.
%   - "\\" (which is usually called to produce a line break) disappears in PDF links.
%   - If you use \urlhack{X} in the URL, it is typeset as X, but it disappears
%     in PDF links.
%   - If you call a custom macro in the URL, you can modify its definition
%     for typesetting in \urlprefix and for PDF links by appending to \urlplainascii.

% Typeset a clickable URL
% \url{http://example.com/}
\def\url{\begingroup\begingroup\allowurlchars\urlaux}
\def\urlaux#1{\urlauxarg{#1}\linkurlauxB{\displayurl}}

% Typeset a clickable link to the given URL
% \linkurl{http://example.com/}{text}
\def\linkurl{\begingroup\begingroup\allowurlchars\linkurlaux}
\def\linkurlaux#1{\urlauxarg{#1}\linkurlauxB}
\def\linkurlauxB#1{%
	\leavevmode
	\ifclickable
		{%
			\urlplainascii
			\pdfstartlink\commonlinkargs user {/Subtype/Link /A << /Type/Action /S/URI /URI(\tmpb) >>}\relax
		}%
	\fi
	#1%
	\ifclickable
		\pdfendlink
	\fi
	\endgroup		% opened in \url or \linkurl
}

% Catcode '%' and '#' to 'other'
\def\allowurlchars{\catcode`\%=12\catcode`\#=12\relax}

\def\urlauxarg#1{%
	\endgroup		% opened in \url or \linkurl
	\toks0={#1}\edef\tmpb{\the\toks0}%
	\replacestrings{//}{\urlslashslash}%
	\replacestrings{_}{\urlunderscore}%
	\replacestrings{~}{\urltilde}%
	\replacestrings{/}{\urlslash}%
	\replacestrings{?}{\urlquestion}%
	\replacestrings{&}{\urlamp}%
	\replacestrings{=}{\urlequal}%
	\replacestrings{-}{\urlminus}%
}

% Style switches at the beginning/end of an URL (feel free to re-define them)
\let\urlprefix\it
\let\urlsuffix\/

% Default appearance of characters special in URLs
\def\urlslashslash{/\kern\urlinterslashkern/}
\def\urlunderscore{\_}
\def\urltilde{{\tt\char126}}
\def\urlslash{/\penalty100\relax}
\def\urlquestion{?\penalty100\relax}
\def\urlamp{\&\penalty100\relax}
\def\urlequal{=\penalty100\relax}
\def\urlminus{\hbox{-}}
\def\urlhack#1{#1}

% Kern to place between "//" in an URL
\newdimen\urlinterslashkern
\urlinterslashkern=-0.1em

% Switch auxiliary macros, so special characters expand to plain ASCII characters
% (since we need to replace them in the expand processor, we cannot use \let for that)
% If you want to modify expansion of your macros, extend \urlplainascii using \appendef.
{
\lccode`A=`\_
\lccode`B=`\~
\lccode`C=`\&
\lccode`D=`\%
\lccode`E=`\#
\lowercase{\gdef\urlplainascii{%
	\def\urlslashslash{//}%
	\def\urlunderscore{A}%
	\def\urltilde{B}%
	\def\urlslash{/}%
	\def\urlquestion{?}%
	\def\urlamp{C}%
	\def\urlequal{=}%
	\def\urlminus{-}%
	\def\%{D}%
	\def\#{E}%
	\def\\{}%
	\def\urlhack##1{}%
}}}

% Typeset the URL stored in \tmpb. In most cases, this is used internally by \url,
% but you can call it explicity from the second argument of \linkurl to typeset the current URL.
\def\displayurl{{\urlprefix\tmpb\urlsuffix}}
