\newif\ifdoublecolumn
\doublecolumnfalse

\input makra-cv

\newdimen\hei
\hei=3cm
\line{%
 \vbox{\vss\hbox{\fnad \myname\ }}
 \hfil
 \hfil
 \hfil
 \vbox{%
  \vss
  \g[img/home.pdf: \myaddress]
  \g[img/phone.pdf: \myphone]
  \vskip -.5mm
  \vss
 }
 \hfil
 \vbox{%
  \vss
  \g[img/mail.pdf: \myemail]
  \g[img/www.pdf: \url{http://\myweb}{\myweb}]
  \vskip -.5mm
  \vss
 }%
}

\vskip 2mm
\hrule width \hsize

\vskip 3mm

{\I
%
A first-year student of the Computer science MSc programme at
ETH Zürich, currently exploring ways of maximising my impact.
%
Open source enthusiast and an aspiring Effective Altruist.
%
I love pushing the boundaries of human knowledge and creating new and
meaningful things, especially in a group of similarly passionate people.
%
Research interests: algorithms, data structures, combinatorial
optimization and flow networks.}

\sekce Education

\table{
Sept.~2021--Now & MSc in Computer science\hfil{\I (expected graduation: June 2023)}\par{\bf ETH Zürich}\bigstrut\cr

 Sept.~2017--June~2021 & Bc. in Computer science\par{\bf Charles University}, Faculty of Mathematics and Physics, Prague\par
 {\I perfect grades (GPA of 1.00) throughout the studies and 224 ECTS credits}\bigstrut\cr
 May 2017 & upper secondary education\par {\bf Grammar School and Commercial Academy Mariánské Lázně}\cr
}

\sekce Achievements \& Awards

\footnote{}{\hskip -3mm\hbox to 3mm{$^\dagger$\hss}a team competition; with my friends Václav Volhejn and Filip Bialas}

\table{
	2021 & ETH-D scholarship for excellent Master's students\cr
%	2019 & ACM-ICPC Central European Regional Contest\global\footnote{$^\dagger$}{with V.~Volhejn and F.~Bialas} – \url{https://icpc.baylor.edu/regionals/finder/central-europe-2019/standings}{5th place}, advancing to World Finals 2020\cr
%	2018 & ACM-ICPC Central European Regional Contest$^\dagger$ – \url{https://contest.felk.cvut.cz/18cerc/rank.html}{9th place}\cr
	2018 & ACM-ICPC World Finals$^\dagger$ – \url{https://icpc.global/regionals/finder/world-finals-2018/standings}{56th place}\cr
	2017 & International Olympiad in Informatics – \url{http://stats.ioinformatics.org/results/2017}{Silver medal (69th place)}\cr
%	2017 & ACM-ICPC Central European Regional Contest$^\dagger$ – \url{http://cerc.hsin.hr/index.php?page=results}{6th place}\cr
%	2016 & Czech Olympiad in Informatics – \url{http://mo.mff.cuni.cz/p/65/vysledky-3.html}{3rd place}\cr
	2016 & ACM-ICPC Central European Regional Contest$^\dagger$ – \url{https://icpc.baylor.edu/regionals/finder/central-europe-2016/standings}{12th place} (unofficial high school participation)\cr
	2016 & International Olympiad in Informatics – \url{http://stats.ioinformatics.org/results/2016}{Bronze medal (154th place)}\cr
}

\sekce Research Experience

\def\hs{\advance\hsize by -4.05cm}
\table{
Mar.~2020–Jun.~2021 &
\hs
{\I \url{https://www.mff.cuni.cz/en/iuuk}{Computer Science Institute}, Charles
University, Prague.} Research in network flows and convex optimisation, supervised by
\url{http://research.koutecky.name/}{Martin Koutecký}. A part
of a Student Faculty Grant and a superset of my bachelor's thesis.
%
We have proposed two new combinatorial algorithms for the multicommodity flow (MCF) problem, and tied their time complexity to combinatorial properties of the MCF polyhedron.
%
Journal article(s) in progress.
\bigstrut
\cr
June 2018–Dec.~2020 &
\hs
{\I \url{http://industrialinformatics.fel.cvut.cz}{Industrial Informatics Research Center}, Czech Technical University, Prague.}
%
Part-ti\-me research in scheduling and optimisation, supervised by Anna Minaeva.
%
Proved NP-hardness of a certain periodic scheduling problem, developed and tested several new heuristics for it, proposed a new method of instance generation. The best heuristic solves even
moderately-sized instances optimally or almost optimally in a matter of seconds.
\bigstrut
\cr
}

\sekce Publications

\unskip
\nocite{hladik2020complexity}%
\bibliography{bibliography}%
\bibliographystyle{plain}%

\sekce Projects

\nproj{\url{http://uralyx.cz/prog/outotune}{Outotune}}{2020–Now}
%
a \url{https://www.youtube.com/watch?v=DnpVAyPjxDA}{harmoniser} with
VST3, LV2 and JACK support written in \Cpp. Lets one sing chords in
real time using a MIDI keyboard and one's voice, by analysing it and
resynthesising it at different pitches.

\sekce Programming languages \& technologies
 
{\let\bf\relax
Professionally worked with {\bf \Cpp} and {\bf Python}, also fluent in {\bf C} and
{\bf sh}. Worked with \Cis{} and Haskell at some point. Experience with
{\bf Git}, {\bf NumPy}, {\bf PyTorch}, \TeX, {\bf TensorFlow}, {\bf Pandas}, among
others. Long-time Linux user with server administration experience.
}

\iffalse
\sekce Teaching

\itemindent=.5\itemindent
\itemnarrow=0pt
\list{-}
	\: Programming II practicals for advanced students (\url{https://mj.ucw.cz/vyuka/1920/p2x/}{Spring~2020}, \url{https://mj.ucw.cz/vyuka/1819/p2x/}{Spring~2019}), co-taught with Martin Mareš
	\: Programming I practicals for advanced students (\url{https://mj.ucw.cz/vyuka/1920/p1x/}{Autumn~2019}), co-taught with Martin Mareš
\endlist

\fi

\sekce Other

{\bf Languages}: Czech (native), English (C2 -- CAE Grade A), German (B2), French (basics)

{\bf Authorisations}: I'm a holder of an EU passport and an EU driver's licence of class B.

\medskip
In my free time, I help to organise seminars, competitions and
educational camps for talented highschoolers:
\medskip

\def\hs{\advance\hsize by -2.1cm}
\table{
2018--Now &\hs {\sl \url{https://mo.mff.cuni.cz/p/}{Czech Olympiad in Informatics}}, {\sl Czech IOI Selection Camp} -- a series of off- and on-site competitions for Czech highschoolers that selects the Czech team for the International Olympiad in Informatics. \bigstrut\cr
%
2017--2020 &\hs {\sl \url{https://ksp.mff.cuni.cz}{KSP}} -- an algorithmisation correspondence seminar for Czech highschoolers; had one
of leading roles in 2018–2019, managing 10–20 people and having responsibility for the seminar's main
category%
.
%– from inventing problems and writing problem statements to testing them and grading contestants' submissions.
\bigstrut\cr
%
2018--2019 &\hs {\sl Czech-Polish-Slovak Preparation Camp} -- a series of on-site competitions for high school students advancing to the International Olympiad in Informatics.\bigstrut\cr
%
%2017--2019 &\hs {\sl \url{https://kasiopea.matfyz.cz}{Kasiopea}}\bigstrut\cr
}

%\sekce Projects
%
%\proj{\url{http://uralyx.cz/prog/outotune}{Outotune}}{real-time harmoniser}{2020}
%
%A \url{https://www.youtube.com/watch?v=DnpVAyPjxDA}{harmoniser} plugin with
%VST3, LV2 and JACK integration written in \Cpp. Lets you sing harmonies in
%real time using only your voice and a MIDI keyboard. Analyses your voice and
%synthesizes it at different pitches.
%
%\proj{\url{https://github.com/RichardHladik/rambajz}{Rambajz}}{real-time tuner
%and spectogram}{2019}
%
%Graphical real-time tuner and infinitely zoomable spectogram.
%Written in C, using SDL and JACK, with emphasis on low latency and high
%tuning precision.
%
%% \proj{\url{https://github.com/RichardHladik/haskell-segtree}{haskell-segtree}}{polymorphic lazy segment tree}{2018}
%% 
%% Generic implementation of a persistent sparse lazy segment tree, which is able
%% to work with arbitrary monoids. Written in Haskell.
%
%\proj{\url{https://github.com/trinerdi/icpc-notebook}{icpc-notebook}}{collection of algorithms and data structures}{2016–2018}
%
%A collection of algorithms and data structures which was used by our team at
%the ACM ICPC World Finals 2018.

\bye
