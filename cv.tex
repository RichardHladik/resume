\newif\ifdoublecolumn
\doublecolumnfalse

\input makra-cv

\let\B\bf
\let\af\relax

\newdimen\hei
\hei=3cm
\line{%
 \vbox{\vss\hbox{\fnad \myname\ }}
 \hfil
 \hfil
 \hfil
 \vbox{%
  \vss
  \g[img/home.pdf: \myaddress]
  \g[img/born.pdf: \mybirth]
  \vskip -.5mm
  \vss
 }
 \hfil
 \vbox{%
  \vss
  \g[img/mail.pdf: \myemail]
  \g[img/phone.pdf: \myphone]
  \vskip -.5mm
  \vss
 }
 \hfil
 \vbox{%
  \vss
  \g[img/github.pdf: \url{https://github.com/\mygithub}{\mygithub}]
  \g[img/www.pdf: \url{http://\myweb}{\myweb}]
  \vskip -.5mm
  \vss
 }%
}

\vskip 2mm
\hrule width \hsize

\vskip 3mm

{\I
%
A second-semester student of the Computer science MSc programme at ETH Zürich.
Theoretical computer scientist at heart, but currently also trying to gain
experience in more practical fields, such as ML, bioinformatics, operation
research or software engineering.
%
Open source enthusiast.
%
I love pushing the boundaries of human knowledge and creating new and
meaningful things, especially in a group of similarly passionate people.
}

{\bf Programming languages:} {\af advanced:} {\B Python}, {\B \Cpp}, {\B C}, {\B sh}; {\af intermediate:} {\B SQL}; {\af basic:} \Cis{}, Haskell.


\sekce Education

\def\hs{\advance\hsize by -2.45cm}

\table{
Sept.~2021--Now &\hs MSc in Computer science, {\bf ETH Zürich}\hfil{\I (expected graduation: March 2024)}\par
{\I GPA of 5.58 (out of 6.00) after the first semester} \bigstrut\cr

 Sept.~2017--June~2021 & Bc.~in Computer science, {\bf Charles University}, \iffalse Faculty of Mathematics and Physics,\fi Prague\par
 {\I perfect grades (GPA of 1.00) throughout the studies and 224 ECTS credits}\cr
% May 2017 & Matura, Grammar School and Commercial Academy Mariánské Lázně, Czechia\cr
}

\sekce Past Experience

\def\hs{\advance\hsize by -4.05cm}
\table{
Mar.~2020–Jun.~2021 &
\hs
{\bf Algorithm research,}
{\I 
%\url{https://www.mff.cuni.cz/en/iuuk}{Computer Science Institute},
Charles University, Prague.}
%
Proposed two {\bf new combinatorial algorithms} for the multicommodity flow
(MCF) problem and {\bf improved the state of the art} convergence for combinatorial MCF
algorithms from $\O(\varepsilon^{-2})$ to $\O(\log 1/\varepsilon)$ for
instances with small inflation rate.
%
%Both algorithms seem to work well in practice.
%One of the algorithms stems from the Frank-Wolfe method, a projection-free convex optimisation method gaining popularity in ML.
%
{\bf Journal article(s)} in progress.
%A part of a Student Faculty Grant and a
Superset of my \url{http://rihl.uralyx.cz/bc-thesis.pdf}{bachelor's thesis};
supervised by \url{http://research.koutecky.name/}{Martin Koutecký}.
Includes a C++ implemenatation.%
\bigstrut
\cr
June 2018–Dec.~2020 &
\hs
{\bf Optimisation research,}
{\I %\url{http://industrialinformatics.fel.cvut.cz}{Industrial Informatics Research Center},
Czech Technical University, Prague.}
%
%Part-time research,
%supervised by Anna Minaeva.
%
{\bf Proved NP-hardness} of a novel periodic scheduling problem, {\bf
developed} (\Cpp) several {\bf heuristics} for it.
%The best one solves 98.4\% of non-overconstrained instances under $3$ minutes.
{\bf Decreased} the approximation error {\bf 10 times} compared to the base implementation.
\bigstrut\cr
}

\vskip -3mm

\sekce Projects

\nproj{\url{http://uralyx.cz/prog/outotune}{Outotune}}{2020–Now}
%
Implemented a \url{https://www.youtube.com/watch?v=DnpVAyPjxDA}{{\bf
harmoniser}} in \Cpp{}. Along the way, {\bf increased the performance} of an
open-source DSP library {\bf 3 times} by optimising FFT usage and data reuse. A
harmoniser lets you {\bf sing harmonies} in real time by {\bf synthesising} the
chords you play on your keyboard {\bf using your voice.}

\nproj{{Operating Systems project}}{2019}
%
implemented a small MIPS OS in C (including interrupt management, heap allocator,
scheduler, virtual memory etc.) in a two-person team, as a part of a university course.

%\nproj{\url{https://github.com/RichardHladik/rambajz}{Rambajz}}{2019}
%%
%graphical real-time tuner and infinitely zoomable spectrogram.
%Written in C, using SDL and JACK, with emphasis on low latency and high
%tuning precision.

\sekce Skills
 

{\bf Technologies:} {\af Git}, {\af NumPy}, {\af PyTorch}, {\af TensorFlow},
{\af Pandas}. Long-time ($> 10$ years) Linux user with sysadmin and systems
programming experience.

{\bf Relevant courses:} Computer Linguistics, Deep Learning, Computer Graphics,
Data Compression Algorithms, Reliable and Trustworthy AI, Information
Theory. Currently taking Principles of Distributed Computing.

\iffalse
\sekce Teaching

\itemindent=.5\itemindent
\itemnarrow=0pt
\list{-}
	\: Programming II practicals for advanced students (\url{https://mj.ucw.cz/vyuka/1920/p2x/}{Spring~2020}, \url{https://mj.ucw.cz/vyuka/1819/p2x/}{Spring~2019}), co-taught with Martin Mareš
	\: Programming I practicals for advanced students (\url{https://mj.ucw.cz/vyuka/1920/p1x/}{Autumn~2019}), co-taught with Martin Mareš
\endlist

\fi

{\bf Languages}: Czech (native), English (C2 -- CAE Grade A), German (B2), French (basics)

\sekce Extracurricular activities

%{\bf Authorisations}: I'm a holder of an EU passport and an EU driver's licence of class B.

%\medskip
%In my free time, I help to organise seminars, competitions and educational camps for talented highschoolers:

\def\hs{\advance\hsize by -2.1cm}
\table{
2018--Now &\hs {\sl \url{https://mo.mff.cuni.cz/p/}{Czech Olympiad in Informatics}}, {\sl Czech IOI Selection Camp} –
I implement, test and write up algorithmic tasks; in the latter, I {\bf reimplemented} and {\bf simplified} a big part of the technical infrastructure\bigstrut\cr
%-- a series of off- and on-site competitions for Czech highschoolers that selects the Czech team for the International Olympiad in Informatics. \bigstrut\cr
%
2017--2020 &\hs {\sl \url{https://ksp.mff.cuni.cz}{KSP}} -- an algorithmisation correspondence seminar for Czech highschoolers; {\bf led the main category} in 2018–2019, managing 10–20 people%
%– from inventing problems and writing problem statements to testing them and grading contestants' submissions.
\bigstrut\cr
%
%2018--2019 &\hs {\sl Czech-Polish-Slovak Preparation Camp} -- a series of on-site competitions for high school students advancing to the International Olympiad in Informatics.\bigstrut\cr
%
%2017--2019 &\hs {\sl \url{https://kasiopea.matfyz.cz}{Kasiopea}}\bigstrut\cr
}

%% \proj{\url{https://github.com/RichardHladik/haskell-segtree}{haskell-segtree}}{polymorphic lazy segment tree}{2018}
%% 
%% Generic implementation of a persistent sparse lazy segment tree, which is able
%% to work with arbitrary monoids. Written in Haskell.
%
%\proj{\url{https://github.com/trinerdi/icpc-notebook}{icpc-notebook}}{collection of algorithms and data structures}{2016–2018}
%
%A collection of algorithms and data structures which was used by our team at
%the ACM ICPC World Finals 2018.

\sekce Achievements \& Awards

\def\footnote#1#2{}
\footnote{}{\hskip -3mm\hbox to 3mm{$^\dagger$\hss}a team competition; with my friends Václav Volhejn and Filip Bialas}
\def\fn{$^\dagger$}
\let\fn\relax

\table{
	2021 & ETH-D scholarship for excellent Master's students (\approx{} 90 out of 2500 Master's students awarded each year)\cr
%	2019 & ACM-ICPC Central European Regional Contest\fn{} – \url{https://icpc.baylor.edu/regionals/finder/central-europe-2019/standings}{5th place}, advancing to World Finals 2020\cr
%	2018 & ACM-ICPC Central European Regional Contest\fn{} – \url{https://contest.felk.cvut.cz/18cerc/rank.html}{9th place}\cr
	2018 & ACM-ICPC World Finals\fn{} – \url{https://icpc.global/regionals/finder/world-finals-2018/standings}{56th place} out of 280 teams\cr
	2017 & International Olympiad in Informatics (IOI) – \url{http://stats.ioinformatics.org/results/2017}{Silver medal, 69th place} out of 304 participants\cr
%	2017 & ACM-ICPC Central European Regional Contest\fn{} – \url{http://cerc.hsin.hr/index.php?page=results}{6th place}\cr
%	2016 & Czech Olympiad in Informatics – \url{http://mo.mff.cuni.cz/p/65/vysledky-3.html}{3rd place}\cr
	2016 & ACM-ICPC Central European Regional Contest\fn{} – \url{https://icpc.baylor.edu/regionals/finder/central-europe-2016/standings}{12th place} (unofficial high school participation)\cr
	2016 & International Olympiad in Informatics (IOI) – \url{http://stats.ioinformatics.org/results/2016}{Bronze medal, 154th place} out of 308 participants\cr
}


\sekce Publications

\unskip
\nocite{hladik2020complexity}%
\bibliography{bibliography}%
\bibliographystyle{plain}%

\bye
