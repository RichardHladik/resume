\input makra

\newif\ifphoto
\phototrue
\let\B\bf
\let\af\relax

\line{%
\ifphoto
\advance\hsize by -3.5cm
\fi
\vbox{%
\newdimen\hei
\hei=3cm
\line{%
 \vbox{\vss\hbox{\fnad \myname\ }\vskip 1mm}
 \hfil
 \hfil
 \hfil
 \vbox{%
  \vss
  \g[img/home.pdf: \myaddress]
  \g[img/born.pdf: \mybirth]
  \g[img/mail.pdf: \myemail]
  \vskip -.5mm
  \vss
 }
 \hfil
 \vbox{%
  \vss
  \g[img/phone.pdf: \myphone]
  \g[img/github.pdf: \url{https://github.com/\mygithub}{\mygithub}]
  \g[img/www.pdf: \url{http://\myweb}{\myweb}]
  \vskip -.5mm
  \vss
 }
}

\vskip 2mm
\hrule width \hsize

\vskip 2mm

{\I
%
A final-semester student of the Computer Science MSc programme at ETH Zürich. Currently writing my Master's thesis under Bernhard Haeupler. I'm passionate about graph theory, algorithms, and data structures, but I've been exploring other areas as well.
%
%Open source enthusiast.
%
I love tackling interesting problems and pushing the boundaries of human
knowledge, especially in a group of similarly passionate people. }
}%
\hss
\ifphoto
\vbox{%
 \vss
 \img{width 3cm}{img/me.jpg}%
 \vss
 \vskip -.5mm
}%
\fi
}

\sekce Education

\def\hs{\advance\hsize by -2.45cm}

\splitcols{3.8cm}{
\def\decora#1{\csc #1}
\lin{Sept.~2021--Now}{MSc in Computer Science, {\bf ETH Zürich}\hfill{\I (expected graduation: summer 2024)}

{\I GPA of 5.60 (out of 6.00) after autumn semester 2023}}

\lin{Sept.~2017--June~2021}{Bc.~in Computer Science, {\bf Charles University}, \iffalse Faculty of Mathematics and Physics,\fi Prague

 {\I perfect GPA (1.00) throughout the studies and 224 ECTS credits (out of 180 required)}}
% May 2017 & Matura, Grammar School and Commercial Academy Mariánské Lázně, Czechia\cr
}


\sekce Work \& Research Experience

\exp{Nov.~2023–Apr.~2024}{Student Researcher}{\url{https://barc.ku.dk/}{BARC}, University of Copenhagen}, supervisor: Rasmus Pagh
\list{o}
\: Multiple projects on {\bf differential privacy} and {\bf instance optimality}.
%\: Proposed a mechanism for privately releasing the minimum spanning tree that is {\bf universally almost-optimal} (gives an optimal expected error on every graph topology, up to a $\O(\log n)$ factor).
%\: We solve the almost 40-years-old open problem of Sorting Under Partial Information, 
\endlist

\exp{Apr.--Oct.~2023}{Research Project}{\I ETH Zürich}, supervisor: Bernhard Haeupler

\list{o}
\: We {\bf designed a heap} with a certain beyond-worst-case property and {\bf proved that Dijkstra's algorithm} using any heap with this property {\bf is universally optimal} (as fast as possible on every graph topology).
\endlist

\exp{Sep.~2022–Feb.~2023}{Software Engineering Intern}{\url{https://www.daedalean.ai/}{Daedalean}, Zürich}

\list{o}
%
%\: Developed a novel tool used to {\bf scale up model evaluation} without human annotation (Rust).
\: {\bf Designed} and {\bf implemented algorithms} for matching model detections with air traffic data based on their movement patterns, thus {\bf scaling up} model evaluation without human annotation.
%\: {\bf Devised} and {\bf analysed a heuristic} for deciding camera visibility of external air traffic.
%
\endlist

\exp{Mar.~2020–Jun.~2021}{Student Researcher}{%
%\url{https://www.mff.cuni.cz/en/iuuk}{Computer Science Institute},
Charles University, Prague}, supervisor: Martin Koutecký

\list{o}
%
%\: Algorithm research in {\bf network flows} and {\bf convex optimisation}. Basis for my \url{http://rihl.uralyx.cz/bc-thesis.pdf}{bachelor's thesis}.
\: {\bf Designed} new combinatorial {\bf algorithms} for the multicommodity flow problem (MCF), polynomial with respect to a certain parametrization.
\: {\bf Showed exponential lower bounds} on the circuits of the MCF linear program and on its fractionality.
\endlist

\exp{June 2018–Dec.~2020}{\bf Student Researcher}{%\url{http://industrialinformatics.fel.cvut.cz}{Industrial Informatics Research Center},
Czech Technical University, Prague}, supervisor: Zdeněk Hanzálek
%
%Part-time research,
%supervised by Anna Minaeva.
%
\list{o}
\: {\bf Proved NP-hardness} of a new periodic scheduling problem and {\bf
developed} several {\bf heuristics} for it.
%\: {\bf Decreased} the approximation error {\bf 10$\times$} compared to the base implementation.

\endlist

%\nproj{Lab Project}{Mar.~2022–July 2022}, {\I ETH Zürich} – an optimised implementation of belief propagation.
%\list{o}
%\: A four-person team project focused on {\bf writing fast C code}.
%\: Achieved {\bf 100–500}$\times$ {\bf speedup:} 80–250$\times$ due to optimisations in the underlying algorithm and 1.3–2$\times$ due to vectorisation, better cache locality, memory compaction etc.
%\endlist

\iffalse
\nproj{\url{http://uralyx.cz/prog/outotune}{Outotune}}{Feb.~2020–Now} –
%
a real-time \Cpp{} \url{https://www.youtube.com/watch?v=DnpVAyPjxDA}{{\bf
harmoniser}} implementation.
\list{o}
\: Lets you {\bf sing harmonies} by {\bf synthesising} the chords you play on your keyboard {\bf using your voice.}

\: {\bf Sped up} an open-source DSP library {\bf 3 times} by optimising FFT usage and data reuse.

\endlist
\fi

%\nproj{{Machine Learning projects}}{Mar.~2018–Dec.~2022}, {\I Charles University \& ETH Zürich}
%%
%\list{o}
%\: Various ML projects written in Python (PyTorch and TensorFlow).
%\: Executed neural network {\bf attacks}, {\bf analysis} and {\bf adversarial training} in a 3-person team.
%\: Toy ML projects including image segmentation, PoS tagging, speech recognition, 3D object recognition.
%\endlist

%\nproj{\url{https://github.com/kasiopea-org/pisek}{Písek}}{Mar.–May 2020} – a Python framework for the preparation of programming competitions.
%%
%\list{o}
%\: {\bf Contributed} to the project during the initial phase (was one of the main contributors). I still occasionally help maintain it.
%\: {\bf Added support} for the contest format used at the International Olympiad in Informatics.
%\endlist

%\nproj{{Operating Systems project}}{Sept.~2019–Jan.~2020} – a small MIPS OS in C; part of a university course.
%%
%\list{o}
%\: {\bf Wrote an OS} including interrupt management, heap allocator,
%scheduler, VM etc. in a {\bf two-person team.}
%\endlist

%\nproj{\url{https://github.com/RichardHladik/rambajz}{Rambajz}}{2019}
%%
%graphical real-time tuner and infinitely zoomable spectrogram.
%Written in C, using SDL and JACK, with emphasis on low latency and high
%tuning precision.

%% \proj{\url{https://github.com/RichardHladik/haskell-segtree}{haskell-segtree}}{polymorphic lazy segment tree}{2018}
%% 
%% Generic implementation of a persistent sparse lazy segment tree, which is able
%% to work with arbitrary monoids. Written in Haskell.
%
%\proj{\url{https://github.com/trinerdi/icpc-notebook}{icpc-notebook}}{collection of algorithms and data structures}{2016–2018}
%
%A collection of algorithms and data structures which was used by our team at
%the ACM ICPC World Finals 2018.


 \iffalse
\sekce Skills
 
{\bf Programming languages:} {\af advanced:} {\B Python}, {\B \Cpp}, {\B C}, {\B sh}; {\af intermediate:} {\B Rust}; {\af basic:} \Cis{}, Haskell.

\smallskip

{\bf Technologies:} {\af Git}, {\af NumPy}, {\af PyTorch}, {\af TensorFlow},
{\af Pandas}, {\af GoogleTest}, {\af GDB}, {\af Make}. Long-time ($> 10$ years) Linux user
with sysadmin and {\bf systems programming} experience.

\smallskip

I also try to broaden my scope by taking {\bf courses in related areas}. So
far, I have taken, for example: Computer Linguistics, Deep Learning, Computer Graphics, Data
Compression Algorithms, Reliable and Trustworthy AI, Information Theory,
Principles of Distributed Computing.

\smallskip


{\bf Languages}: Czech (native), English (C2 -- CAE Grade A), German (B2–C1), French (basics)

\fi

\iffalse
\sekce Teaching

\itemindent=.5\itemindent
\itemnarrow=0pt
\list{-}
	\: Programming I/II practicals for advanced students (), co-taught with Martin Mareš
	\: Programming I practicals for advanced students (), co-taught with Martin Mareš
\endlist

\fi

\sekce Teaching \& Extracurricular Activities

%{\bf Authorisations}: I'm a holder of an EU passport and an EU driver's licence of class B.

%\medskip
%In my free time, I help to organise seminars, competitions and educational camps for talented highschoolers:

\splitcols{1.75cm}{
\lin{2018--2022}{{\sl \url{https://mo.mff.cuni.cz/p/}{Czech Olympiad in Informatics}} \& {\sl Czech IOI Selection Camp} –
I proposed and prepared problems, graded solutions and generally helped with the organization.
}
\lin{2017--2023}{{\sl \url{https://ksp.mff.cuni.cz}{KSP}} -- an algorithmic seminar for Czech highschoolers; {\bf main organizer of the main category} in 2018–2019, managing 10–20 organizers. Co-organised educational camps and {\bf gave lectures}.
%with informatics-related (and -unrelated) activities; I {\bf gave lectures} there.
%I also proposed algorithmic problems and did almost everything described in the previous paragraph.
}
\lin{2019--2020}{TA of Programming for advanced students; {\I Charles University} (\url{https://mj.ucw.cz/vyuka/1819/p2x/}{Spring} \& \url{https://mj.ucw.cz/vyuka/1920/p1x/}{Fall} 2019, \url{https://mj.ucw.cz/vyuka/1920/p2x/}{Spring~2020}).}
%Set, discussed and solved algorithmic questions during class and
%I also prepared \Cpp{} programming tasks for the students to solve.}
%
%2018--2019 &\hs {\sl Czech-Polish-Slovak Preparation Camp} -- a series of on-site competitions for high school students advancing to the International Olympiad in Informatics.\bigstrut\cr
%
%2017--2019 &\hs {\sl \url{https://kasiopea.matfyz.cz}{Kasiopea}}\bigstrut\cr
}

\sekce Achievements \& Awards

\def\footnote#1#2{}
\footnote{}{\hskip -3mm\hbox to 3mm{$^\dagger$\hss}a team competition; with my friends Václav Volhejn and Filip Bialas}
\def\fn{$^\dagger$}
\let\fn\relax

\splitcols{8mm}{
	\splitskip=.1\baselineskip
	\lin{2021}{{\bf ETH-D scholarship} for excellent Master's students (\approx{} 90 out of 2500 students awarded each year)}
	\lin{2021}{ACM-ICPC World Finals 2020 – unofficial participation due to COVID}
	\lin{2018}{{\bf ACM-ICPC World Finals}\fn{} – \url{https://icpc.global/regionals/finder/world-finals-2018/standings}{56th place} out of 140 teams}
	\lin{2017}{International Olympiad in Informatics ({\bf IOI}) – \url{http://stats.ioinformatics.org/results/2017}{{\bf Silver medal}, 69th place} out of 304 participants}
%	\lin{2016}{ACM-ICPC Central European Regional Contest\fn{} – \url{https://icpc.global/regionals/finder/central-europe-2016/standings}{12th place} (unofficial high school participation)}
	\lin{2016}{International Olympiad in Informatics (IOI) – \url{http://stats.ioinformatics.org/results/2016}{Bronze medal, 154th place} out of 308 participants}
}


\sekce Publications

\unskip
\nocite{hladik2020complexity}%
\nocite{sssp}%
\nocite{hladik-bp}%
\nocite{supi}%
\nocite{dp-mst}%
\bibliography{bibliography}%
\bibliographystyle{plainyr}%

\bye
