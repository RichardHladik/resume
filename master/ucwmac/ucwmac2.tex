% The UCW Macro Collection (a successor of mjmac.tex)
% Written by Martin Mares <mj@ucw.cz> in 2010--2018 and placed into public domain
% -------------------------------------------------------------------------------

\ifx\ucwmodule\undefined\else\endinput\fi

%%% Prolog %%%

% We'll use internal macros of plain TeX
\catcode`@=11

\ifx\eTeXversion\undefined
\errmessage{ucwmac requires the e-TeX engine or its successor}
\fi

%%% PDF output detection %%%

\newif\ifpdf
\pdffalse

\ifx\pdfoutput\undefined
\else\ifnum\pdfoutput>0
	\pdftrue
	\pdfpkresolution=600	% Provide a reasonable default
\fi\fi

\ifx\directlua\undefined\else
	% In LuaTeX \pdfpkresolution is not enough
	\directlua{kpse.init_prog("luatex", 600, "ljfour")}
\fi

%%% Temporary registers %%%

\newcount\tmpcount
\newdimen\tmpdimen

%%% Auxiliary macros %%%

% Prepend/append #2 to the definition of #1
\long\def\prependef#1#2{\expandafter\def\expandafter#1\expandafter{#2#1}}
\long\def\appendef#1#2{\expandafter\def\expandafter#1\expandafter{#1#2}}

% Variants of \def and \let, where the control sequence name is given as a string
\def\sdef#1{\expandafter\def\csname#1\endcsname}
\def\slet#1#2{\expandafter\let\csname#1\expandafter\endcsname\csname#2\endcsname}

% Assign a control sequence given as a string, complain if it is not defined.
\def\sget#1#2{\ifcsname#2\endcsname
		\expandafter\let\expandafter#1\csname#2\endcsname
	\else
		\errmessage{Undefined control sequence #2}%
		\let#1\relax
	\fi
}

% Add \protected to an existing macro
\def\addprotected#1{\protected\edef#1{\expandafter\unexpanded\expandafter{#1}}}

% Protect ~
\addprotected~

\def\ucwwarn#1{\immediate\write16{*** UCWmac warning: #1 ***}}

%%% Page size and margins %%%

% If you modify these registers, call \setuppage afterwards
\ifx\luatexversion\undefined
	% In LuaTeX, \pagewidth and \pageheight are primitive
	\newdimen\pagewidth
	\newdimen\pageheight
\fi
\newdimen\leftmargin
\newdimen\rightmargin
\newdimen\topmargin
\newdimen\bottommargin
\newdimen\evenpageshift

\def\setuppage{%
	\hsize=\pagewidth
	\advance\hsize by -\leftmargin
	\advance\hsize by -\rightmargin
	\vsize=\pageheight
	\advance\vsize by -\topmargin
	\advance\vsize by -\bottommargin
	\hoffset=\leftmargin
	\advance\hoffset by -1truein
	\voffset=\topmargin
	\advance\voffset by -1truein
	\ifpdf
		\pdfhorigin=1truein
		\pdfvorigin=1truein
		\ifx\luatexversion\undefined
			\pdfpagewidth=\pagewidth
			\pdfpageheight=\pageheight
		\fi
	\fi
}

% Set multiple margins to the same value
\def\sethmargins#1{\leftmargin=#1\relax\rightmargin=#1\relax\evenpageshift=0pt\relax}
\def\setvmargins#1{\topmargin=#1\relax\bottommargin=#1\relax}
\def\setmargins#1{\sethmargins{#1}\setvmargins{#1}}

% Define inner/outer margin instead of left/right
\def\setinneroutermargin#1#2{\leftmargin#1\relax\rightmargin#2\relax\evenpageshift=\rightmargin\advance\evenpageshift by -\leftmargin}

% Use a predefined paper format, calls \setuppage automagically
\def\setpaper#1{%
	\expandafter\let\expandafter\currentpaper\csname paper-#1\endcsname
	\ifx\currentpaper\relax
		\errmessage{Undefined paper format #1}
	\fi
	\currentpaper
}

% Switch to landscape orientation, calls \setuppage automagically
\def\landscape{%
	\dimen0=\pageheight
	\pageheight=\pagewidth
	\pagewidth=\dimen0
	\setuppage
}

% Common paper sizes
\def\defpaper#1#2#3{\expandafter\def\csname paper-#1\endcsname{\pagewidth=#2\pageheight=#3\setuppage}}
\defpaper{a3}{297truemm}{420truemm}
\defpaper{a4}{210truemm}{297truemm}
\defpaper{a5}{148truemm}{210truemm}
\defpaper{letter}{8.5truein}{11truein}
\defpaper{legal}{8.5truein}{14truein}

% Default page parameters
\setmargins{1truein}
\setpaper{a4}

%%% Macros with optional arguments %%%

% After \def\a{\withoptarg\b}, the macro \a behaves in this way:
%	\a[arg]		does \def\optarg{arg} and then it expands \b
%	\a		does \let\optarg=\relax and then it expands \b
\def\withoptarg#1{\let\xoptcall=#1\futurelet\next\xopt}
\def\xopt{\ifx\next[\expandafter\xoptwith\else\let\optarg=\relax\expandafter\xoptcall\fi}
\def\xoptwith[#1]{\def\optarg{#1}\xoptcall}

% A shortcut for defining macros with optional arguments:
% \optdef\macro behaves as \def\domacro, while \macro itself is defined
% as a wrapper calling \domacro using \withoptarg.
\def\optdef#1{%
	\edef\xoptname{\expandafter\eatbackslash\string#1}%
	\edef#1{\noexpand\withoptarg\csname do\xoptname\endcsname}%
	\expandafter\def\csname do\xoptname\endcsname
}

% Trick: \eatbackslash eats the next backslash of category 12
\begingroup\lccode`\+=`\\
\lowercase{\endgroup\def\eatbackslash+{}}

% Expand to the optional argument if it exists
\def\optargorempty{\ifx\optarg\relax\else\optarg\fi}

%%% Placing material at specified coordinates %%%

% Set all dimensions of a given box register to zero
\def\smashbox#1{\ht#1=0pt \dp#1=0pt \wd#1=0pt}
\long\def\smashedhbox#1{{\setbox0=\hbox{#1}\smashbox0\box0}}
\long\def\smashedvbox#1{{\setbox0=\vbox{#1}\smashbox0\box0}}

% Variants of \llap and \rlap working equally on both sides and/or vertically
\def\hlap#1{\hbox to 0pt{\hss #1\hss}}
\def\vlap#1{\vbox to 0pt{\vss #1\vss}}
\def\clap#1{\vlap{\hlap{#1}}}

% \placeat{right}{down}{hmaterial} places <hmaterial>, so that its
% reference point lies at the given position wrt. the current ref point
\long\def\placeat#1#2#3{\smashedhbox{\hskip #1\lower #2\hbox{#3}}}

% Like \vbox, but with reference point in the upper left corner
\long\def\vhang#1{\vtop{\hrule height 0pt\relax #1}}

% Like \vhang, but respecting interline skips
\long\def\vhanglines#1{\vtop{\hbox to 0pt{}#1}}

% Crosshair with reference point in its center
\def\crosshair#1{\clap{\vrule height 0.2pt width #1}\clap{\vrule height #1 width 0.2pt}}

%%% Output routine %%%

\newbox\pageunderlays
\newbox\pageoverlays
\newbox\commonunderlays
\newbox\commonoverlays

% In addition to the normal page contents, you can define page overlays
% and underlays, which are zero-size vboxes positioned absolutely in the
% front / in the back of the normal material. Also, there are global
% versions of both which are not reset after every page.
\def\addlay#1#2{\setbox#1=\vbox{\ifvbox#1\box#1\fi\nointerlineskip\smashedvbox{#2}}}
\def\pageunderlay{\addlay\pageunderlays}
\def\pageoverlay{\addlay\pageoverlays}
\def\commonunderlay{\addlay\commonoverlays}
\def\commonoverlay{\addlay\commonoverlays}

% Our variation on \plainoutput, which manages inner/outer margins and overlays
\output{\ucwoutput}
\newdimen\pagebodydepth
\def\ucwoutput{\wigglepage\shipout\vbox{%
	\makeheadline
	\ifvbox\commonunderlays\copy\commonunderlays\nointerlineskip\fi
	\ifvbox\pageunderlays\box\pageunderlays\nointerlineskip\fi
	\pagebody
	\pagebodydepth=\prevdepth
	\nointerlineskip
	\ifvbox\commonoverlays\vbox to 0pt{\vskip -\vsize\copy\commonoverlays\vss}\nointerlineskip\fi
	\ifvbox\pageoverlays\vbox to 0pt{\vskip -\vsize\box\pageoverlays\vss}\nointerlineskip\fi
	\prevdepth=\pagebodydepth
	\makefootline
}\advancepageno
\ifnum\outputpenalty>-\@MM \else\dosupereject\fi}

\def\wigglepage{\ifodd\pageno\else\advance\hoffset by \evenpageshift\fi}

% Make it easier to redefine footline font (also, fix it so that OFS won't change it unless asked)
\let\footfont=\tenrm
\footline={\hss\footfont\folio\hss}

%%% Itemization %%%

% Usage:
%
% \list{style}
% \:first item
% \:second item
% \endlist
%
% Available styles (others can be defined by \sdef{item:<style>}{<marker>})
%
%	o		% bullet
%	O		% empty circle
%	*		% asterisk
%	-		% en-dash
%	.		% dot
%	n		% 1, 2, 3
%	i		% i, ii, iii
%	I		% I, II, III
%	a		% a, b, c
%	A		% A, B, C
%	g		% α, β, γ
%
% Meta-styles (can be used to modify an arbitrary style, currently hard-wired)
%
%	#.		% with a dot behind
%	#)		% with a parenthesis behind
%	(#)		% enclosed in parentheses
%
% Historic usage:
%
% \itemize\ibull	% or other marker
% \:first item
% \:second item
% \endlist
%
% \numlist\ndotted	% or other numbering style
% \:first
% \:second
% \endlist

% Default dimensions of itemized lists
\newdimen\itemindent		\itemindent=0.5in
\newdimen\itemnarrow		\itemnarrow=0.5in			% make lines narrower by this amount
\newskip\itemmarkerskip		\itemmarkerskip=0.4em			% between marker and the item
\newskip\preitemizeskip		\preitemizeskip=3pt plus 2pt minus 1pt	% before the list
\newskip\postitemizeskip	\postitemizeskip=3pt plus 2pt minus 1pt	% after the list
\newskip\interitemskip		\interitemskip=2pt plus 1pt minus 0.5pt	% between two items

% Analogues for nested lists
\newdimen\nesteditemindent	\nesteditemindent=0.25in
\newdimen\nesteditemnarrow	\nesteditemnarrow=0.25in
\newskip\prenesteditemizeskip	\prenesteditemizeskip=0pt
\newskip\postnesteditemizeskip	\postnesteditemizeskip=0pt

\newif\ifitems\itemsfalse
\newbox\itembox
\newcount\itemcount

% Penalties
\newcount\preitemizepenalty	\preitemizepenalty=-500
\newcount\postitemizepenalty	\postitemizepenalty=-500

\def\preitemize{
	\ifitems
		\vskip\prenesteditemizeskip
		\advance\leftskip by \nesteditemindent
		\advance\rightskip by \nesteditemnarrow
	\else
		\ifnum\preitemizepenalty=0\else\penalty\preitemizepenalty\fi
		\vskip\preitemizeskip
		\advance\leftskip by \itemindent
		\advance\rightskip by \itemnarrow
	\fi
	\parskip=\interitemskip
}

\def\postitemize{
	\ifitems
		\vskip\postnesteditemizeskip
	\else
		\ifnum\postitemizepenalty=0\else\penalty\postitemizepenalty\fi
		\vskip\postitemizeskip
	\fi
}

\def\inititemize{\begingroup\preitemize\itemstrue\parindent=0pt}

\def\list#1{\inititemize\itemcount=0\liststyle{#1}\let\:=\listitem}
\def\listitem{\par\leavevmode\advance\itemcount by 1
	\llap{\listmarker\hskip\itemmarkerskip}\ignorespaces}

\def\liststyle#1{%
	\edef\markertmp{#1}
	\ifcsname item:\markertmp\endcsname
		\sget\listmarker{item:\markertmp}%
	\else
		\sget\listmarker{metaitem:\markertometa#1^^X}%
		\sget\markerinner{item:\markertoinner#1^^X}%
	\fi
}

\def\markertometa#1{%
	\ifx#1^^X%
	\else
		\ifx#1((%
		\else\ifx#1))%
		\else\ifx#1..%
		\else=%
		\fi\fi\fi
		\expandafter\markertometa
	\fi
}

\def\markertoinner#1{%
	\ifx#1^^X%
	\else
		\ifx#1(%
		\else\ifx#1)%
		\else\ifx#1.%
		\else#1%
		\fi\fi\fi
		\expandafter\markertoinner
	\fi
}

\def\endlist{\par\endgroup\postitemize}

% List styles
\sdef{item:o}{\raise0.2ex\hbox{$\bullet$}}
\sdef{item:O}{\raise0.2ex\hbox{$\circ$}}
\sdef{item:*}{\raise0.2ex\hbox{$\ast$}}
\sdef{item:-}{--}
\sdef{item:.}{\raise0.2ex\hbox{$\cdot$}}
\sdef{item:n}{\the\itemcount}
\sdef{item:i}{\romannumeral\itemcount}
\sdef{item:I}{\uppercase\expandafter{\romannumeral\itemcount}}
\sdef{item:a}{\char\numexpr 96+\itemcount\relax}
\sdef{item:A}{\char\numexpr 64+\itemcount\relax}
\sdef{item:g}{$\ifcase\itemcount\or\alpha\or\beta\or\gamma\or\delta\or\epsilon\or
\zeta\or\eta\or\theta\or\iota\or\kappa\or\lambda\or\mu\or\nu\or\xi\or\pi\or\rho
\or\sigma\or\tau\or\upsilon\or\phi\or\chi\or\psi\or\omega\fi$}

% List meta-styles
\sdef{metaitem:=.}{\markerinner.}
\sdef{metaitem:=)}{\markerinner)}
\sdef{metaitem:(=)}{(\markerinner)}

% Old-style lists

\def\itemize#1{\inititemize\setbox\itembox\llap{#1\hskip\itemmarkerskip}%
\let\:=\singleitem}

\def\singleitem{\par\leavevmode\copy\itembox\ignorespaces}

\def\numlist#1{\inititemize\itemcount=0\let\:=\numbereditem
\let\itemnumbering=#1}

\def\numbereditem{\par\leavevmode\advance\itemcount by 1
\llap{\itemnumbering\hskip\itemmarkerskip}\ignorespaces}

% Old-style markers

\def\ibull{\raise0.2ex\hbox{$\bullet$}}
\def\idot{\raise0.2ex\hbox{$\cdot$}}
\def\istar{\raise0.2ex\hbox{$\ast$}}

\def\nnorm{\the\itemcount}
\def\ndotted{\nnorm.}
\def\nparen{\nnorm)}
\def\nparenp{(\nnorm)}
\def\nroman{\romannumeral\itemcount}
\def\nromanp{\nroman)}
\def\nalpha{\count@=96\advance\count@ by\itemcount\char\count@)}
\def\nAlpha{\count@=64\advance\count@ by\itemcount\char\count@)}
\def\ngreek{$\ifcase\itemcount\or\alpha\or\beta\or\gamma\or\delta\or\epsilon\or
\zeta\or\eta\or\theta\or\iota\or\kappa\or\lambda\or\mu\or\nu\or\xi\or\pi\or\rho
\or\sigma\or\tau\or\upsilon\or\phi\or\chi\or\psi\or\omega\fi$)}

%%% Miscellanea %%%

% {\I italic} with automatic italic correction
\def\I{\it\aftergroup\/}

% A breakable dash, to be repeated on the next line
\def\={\discretionary{-}{-}{-}}

% Non-breakable identifiers
\def\<#1>{\leavevmode\hbox{\I #1}}

% Handy shortcuts
\let\>=\noindent
\def\\{\hfil\break}

% Variants of \centerline, \leftline and \rightline, which are compatible with
% verbatim environments and other catcode hacks
\def\cline{\bgroup\def\linet@mp{\aftergroup\box\aftergroup0\aftergroup\egroup\hss\bgroup\aftergroup\hss\aftergroup\egroup}\afterassignment\linet@mp\setbox0\hbox to \hsize}
\def\lline{\bgroup\def\linet@mp{\aftergroup\box\aftergroup0\aftergroup\egroup\bgroup\aftergroup\hss\aftergroup\egroup}\afterassignment\linet@mp\setbox0\hbox to \hsize}
\def\rline{\bgroup\def\linet@mp{\aftergroup\box\aftergroup0\aftergroup\egroup\hss\bgroup\aftergroup\egroup}\afterassignment\linet@mp\setbox0\hbox to \hsize}

% Insert a PDF picture
% \putimage{width specification}{file}
\def\putimage#1#2{\hbox{\pdfximage #1{#2}\pdfrefximage\pdflastximage}}

%%% Colors %%%

% Use of pdfTeX color stack:
% \colorpush\rgb{1 0 0} puts a new color on the stack
% \colorset\rgb{1 0 0} replaces the top color on the stack
% \colorpop pops the top color
% \colorlocal\rgb{1 0 0} set a color locally until the end of the current group
%\chardef\colorstk=\pdfcolorstackinit page direct{0 g 0 G}
\def\colorset#1{\pdfcolorstack\colorstk set #1}
\def\colorpush#1{\pdfcolorstack\colorstk push #1}
\def\colorpop{\pdfcolorstack\colorstk pop}
\def\colorlocal{\aftergroup\colorpop\colorpush}

% Different ways of describing colors: \rgb{R G B}, \gray{G}, \cmyk{C M Y K}
% (all components are real numbers between 0 and 1)
\def\rgb#1{{#1 rg #1 RG}}
\def\gray#1{{#1 g #1 G}}
\def\cmyk#1{{#1 k #1 K}}

%%% Localization %%%

% Define a new localized string: \localedef{language}{identifier}{message}
% (we use \language codes to identify languages)
\def\localedef#1#2{\tmpcount=#1\expandafter\def\csname loc:\the\tmpcount:#2\endcsname}

% Expand a localized string in the current language: \localemsg{identifier}
\def\localestr#1{%
	\ifcsname loc:\the\language:#1\endcsname
		\csname loc:\the\language:#1\endcsname
	\else
		\ucwwarn{Localized string #1 not defined in language \the\language}%
		???%
	\fi
}

%%% Modules %%%

% Require a module: load it if it is not already loaded
\def\ucwmodule#1{
	\ifcsname ucwmod:#1\endcsname
	\else
		\input ucwmac/ucw-#1.tex
	\fi
}

% Definition of a new module (to be placed at the beginning of its file)
% (Also guards against repeated loading if somebody uses \input instead of \ucwmodule.)
\def\ucwdefmodule#1{
	\ifcsname ucwmod:#1\endcsname\endinput\fi
	\expandafter\let\csname ucwmod:#1\endcsname=\relax
}

%%% Epilog %%%

% Let's hide all internal macros
\catcode`@=12
