\input makra

\hyphenation{pro-ved}
\hyphenation{pre-par-ing}
\hyphenation{zoom-able}

\newdimen\hei
\hei=1.5cm
\doublecolumns
 \vbox to \hei{
 {\vfill\fnad Richard Hladík}}
 \vbox to \hei{\vfill\parskip=0pt
 \g[Email: work@uralyx.cz]
 \g[Telephone: +420 608 176 814]
 \g[Github: \url{https://github.com/RichardHladik/rambajz}{RichardHladik}]}
\endcolumns

\doublecolumns

\sekce Experience

% TODO: better name
\fancy{Optimisation Research
Intern}{\url{http://industrialinformatics.fel.cvut.cz}{Industrial Informatics
Research Center}, Czech Technical University, Prague}{June 2018–present}
Research in scheduling and optimisation. Proved NP-hardness of a certain
periodic scheduling problem, developed a heuristic which solves even
moderate-sized instances efficiently. Currently working on a paper describing
the results.

\sekce Education

\fancy{Charles University}{Prague}{2017–present}

Studying {\bf General computer science} at the Faculty of Mathematics and
Physics. As of the time of writing, perfect grades and 173 ECTS credits in the
first two years.

\sekce Skills, abilities

\f[Programming languages and technologies: professionally worked with {\bf
\Cpp} and {\bf Python}, fluent in {\bf C} and {\bf sh}. Knowledge of \Cis,
Haskell, \TeX, some knowledge of Rust, Perl, HTML+Javascript+CSS. Comfortable
with Git, TensorFlow, NumPy, Pandas.]

Have been actively using {\bf Linux} on both desktop and server for over ten
years, experience with scripting and system administration. Knowledge of
various Linux distributions (from both user and administrator point of view) –
Arch Linux, Debian, Gentoo, Ubuntu, etc.

Interested in {\bf data structures}, {\bf algorithmisation}, discrete
optimisation, and systems programming.

Strong algorithmic thinking proven in programming contests.

\sekce Competitive programming

\url{https://icpc.baylor.edu/regionals/finder/world-finals-2018/standings}{56th
place} at ACM-ICPC World Finals 2018.

\url{https://contest.felk.cvut.cz/18cerc/rank.html}{9th place} in the Central
European Regional Contest (CERC) 2018.
\url{http://cerc.hsin.hr/index.php?page=results}{6th place} in 2017,
\url{https://icpc.baylor.edu/regionals/finder/central-europe-2016/standings}{12th
place} in 2016 (unofficial participation).

\url{http://stats.ioinformatics.org/results/2017}{Silver medal (69th place)} at
the International Olympiad in Informatics 2017,
\url{http://stats.ioinformatics.org/results/2016}{bronze medal (154th place)}
in 2016.

\url{http://mo.mff.cuni.cz/p/65/vysledky-3.html}{3rd place} in the Czech
Mathematical Olympiad (category P) 2016, Czech national pre-IOI competition

\sekce Projects

\proj{\url{https://github.com/RichardHladik/rambajz}{Rambajz}}{real-time tuner
and spectogram}{2019}

Graphical real-time tuner with a built-in infinitely zoomable spectogram.
Written in C, using SDL and JACK, with emphasis on low latency and high
tuning precision.

\proj{\url{https://github.com/RichardHladik/haskell-segtree}{haskell-segtree}}{polymorphic lazy segment tree}{2018}

Generic implementation of a persistent sparse lazy segment tree, which is able
to work with arbitrary monoids. Written in Haskell.

\sekce Languages
\vskip\parskip

\dvoj {\bf Czech} – native! {\bf English} – C1

\dvoj {\bf German} – B2! {\bf French} – basics

\sekce Other activities

Co-organised the {\bf Czech IOI Selection Camp 2018} and {\bf 2019}, a series of
on-site competitions for Czech high school students. Was responsible for preparing
and testing tasks and managing the contest environment.

Co-organised the {\bf Czech-Polish-Slovak Preparation Camp 2018} and {\bf 2019}, a
series of on-site competitions for high school students advancing to the
International Olympiad in Informatics. Was responsible for proposing, preparing
and testing tasks.

Co-organised {\bf \url{https://ksp.mff.cuni.cz}{KSP}} {\sl (2017–2019)}, a
programming correspondence seminar for Czech high school students interested in
algorithmisation and data structures. Had one of leading roles in 2018–2019.

Co-organised {\bf \url{https://kasiopea.matfyz.cz}{Kasiopea}} {\sl (2017–2019)}, a
programming competition for high school students of all levels of programming
skills. Helped prepare and test tasks.

\endcolumns

\bye
